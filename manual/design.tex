\documentclass[a4paper,11pt,twocolumn]{article}
\usepackage{parskip}
\usepackage{amsthm}
\usepackage{amsmath}
\usepackage{amsfonts}
\topmargin -1.5cm
\oddsidemargin -0.04cm
\evensidemargin -0.04cm
\textwidth 16.59cm
\textheight 25cm

\title{
  Design and Implementation of an Interactive Painter\\
}

\author{
  GUO Yuxiang, WU Sisi, ZHANG Yaofeng, ZHANG Yusi, ZHAO Guanlun
}

\begin{document}

\maketitle

\abstract{}

The Interactive Painter is an online application that allows multiple users to
interactively paint on one canvas and communicate with each other at the same
time. Based on the web socket mechanism, this application works quite
efficiently and can deal with relatively large amount of data transfer. The
design and development of this web application is motivated by Google Docs,
which is a convenient platform for collaborations. Since a graphical version
is not available, we started this project to design and implement this
painter.

\section{Basic Idea}

Here we briefly explains how the application works in the perspective of a
user.

When the user enters the website, he/she will be prompted either to join an
existing workspace or create a new one. If the user choose to create a new
workspace, an instance of a workspace will be created in the server side,
otherwise all the existing data will be passed to the client side browser and
the existing canvas will be constructed according to the data from the server
side (This part will be covered shortly).

When the user has entered the workspace, he/she can do the following: to draw
on the canvas, to chat with other painters through messages, to save and
download the work, or many others. These different activities will be
converted into a string that contains the relevant message and passed to the
server through the same interface.

The server receives the data from the clients, parse the string and deal with
the different requests by calling different methods. The methods accept a
string and update the server-side data and then return a new string which will
be sent back to the clients.

After receiving the data the server sends back, the web browser draws on the
canvas and all the users can simultaneously see the updates of other painters.

\section{Framework}

Thanks to the emerging of HTML5 canvas, which is a nice platform for
developing various browser-based graphical applications, our project can start
much easier than in the dark, old days when the similar functionalities have to be
implemented by Flash of SVG. The canvas mechanism provides a lot of handy APIs
that can easily be adopted.

The new HTML5 also provides built-in support for web socket, which is an
efficient and user-friendly communication tool for data transfer. Now the web
socket is directly accessible from Javascript. However, one important drawback
of the build-in web socket APIs is that they are now available for only a few
web browsers, including Google Chrome, Chromium, and Safari, not even
Mozilla Firefox (up till now), since the common agreement of HTML5 standard
has not been reached yet.

The server side is implemented using the Perl web framework of Mojolicious,
which provide some useful APIs for web programming, including web socket.
Mojolicious works quite efficiently when large amount of data need to be
transfered.

\section{Design}

Coming back to the client side, we have to draw the lines and shapes on the
canvas. Since we need to deal with a lot of different situations (covered
later), we apply a special structure to this application. All the users share
a common canvas that contains the "static" part of the canvas, and at the same
time each user has a transparent layer, which is also a canvas, to draw
his/her most recent 



\end{document}
